\documentclass{article}
\usepackage[utf8]{inputenc}
\usepackage{amsmath, amssymb}
\usepackage{setspace}

\onehalfspacing

\title{Problem Set 19: Determinant Formulas and Cofactors}
\author{Tiago C. Botelho}
\date{\today}

\begin{document}

\maketitle

\noindent \textbf{Problem 19.1:} We can perform a bunch of row exchanges to get the identity matrix. But we have to be precise about how much is a bunch in this case:

\[
A = \begin{bmatrix}
0 & 0 & 0 & 1\\
1 & 0 & 0 & 0\\
0 & 1 & 0 & 0\\
0 & 0 & 1 & 0\\
\end{bmatrix}
\to
\begin{bmatrix}
1 & 0 & 0 & 0\\
0 & 0 & 0 & 1\\
0 & 1 & 0 & 0\\
0 & 0 & 1 & 0\\
\end{bmatrix}
\to
\begin{bmatrix}
1 & 0 & 0 & 0\\
0 & 1 & 0 & 0\\
0 & 0 & 0 & 1\\
0 & 0 & 1 & 0\\
\end{bmatrix}
\to
\begin{bmatrix}
1 & 0 & 0 & 0\\
0 & 1 & 0 & 0\\
0 & 0 & 1 & 0\\
0 & 0 & 0 & 1\\
\end{bmatrix}
= I.
\]

OK! Apparently a bunch means 3, an \textit{odd} number of row swaps. This means that the sign of $\det A$ is the opposite of the sign of $\det I$. But when we were defining the properties a determinant must have, one of our building blocks was $\det I = 1$. Since all we did to obtain the identity matrix was an odd number of row exchanges, we get: $\det A = -1$. We prefer this method because it requires a lot less thinking.

\noindent \textbf{Problem 19.2:} Ok, let's use rule 3, which is easier than cofactors. Rule 3 says determinants are linear in each row. So:

\[
\begin{vmatrix}
1 & 1 & 1 & 1\\
1 & 2 & 3 & 4\\
1 & 3 & 6 & 10\\
1 & 4 & 10 & 20\\
\end{vmatrix}
=
20 \begin{vmatrix}
1 & 1 & 1\\
1 & 2 & 3\\
1 & 3 & 6\\
\end{vmatrix}
-10 \begin{vmatrix}
1 & 1 & 1\\
1 & 2 & 4\\
1 & 3 & 10\\
\end{vmatrix}
+4 \begin{vmatrix}
1 & 1 & 1\\
1 & 3 & 4\\
1 & 6 & 10\\
\end{vmatrix}
-\begin{vmatrix}
1 & 1 & 1\\
2 & 3 & 4\\
3 & 6 & 10\\
\end{vmatrix},
\]

whereas:

\[
\begin{vmatrix}
1 & 1 & 1 & 1\\
1 & 2 & 3 & 4\\
1 & 3 & 6 & 10\\
1 & 4 & 10 & 19\\
\end{vmatrix}
=
19 \begin{vmatrix}
1 & 1 & 1\\
1 & 2 & 3\\
1 & 3 & 6\\
\end{vmatrix}
-10 \begin{vmatrix}
1 & 1 & 1\\
1 & 2 & 4\\
1 & 3 & 10\\
\end{vmatrix}
+4 \begin{vmatrix}
1 & 1 & 1\\
1 & 3 & 4\\
1 & 6 & 10\\
\end{vmatrix}
-\begin{vmatrix}
1 & 1 & 1\\
2 & 3 & 4\\
3 & 6 & 10\\
\end{vmatrix}.
\]

We can subtract the second expression from the first one to obtain:

\[
1
-
\begin{vmatrix}
1 & 1 & 1 & 1\\
1 & 2 & 3 & 4\\
1 & 3 & 6 & 10\\
1 & 4 & 10 & 19\\
\end{vmatrix}
=
\begin{vmatrix}
1 & 1 & 1\\
1 & 2 & 3\\
1 & 3 & 6\\
\end{vmatrix}.
\]

Elimination on the right hand side produces an upper triangular matrix without changing the determinant:

\[
\begin{vmatrix}
1 & 1 & 1\\
0 & 1 & 2\\
0 & 0 & 1\\
\end{vmatrix}
= 1,
\]

where we know that this right hand side must equal to 1, for the determinant of upper triangular matrices is just the product of the entries on its diagonal.

But then:

\[
\begin{vmatrix}
1 & 1 & 1 & 1\\
1 & 2 & 3 & 4\\
1 & 3 & 6 & 10\\
1 & 4 & 10 & 19\\
\end{vmatrix}
=
1 - 1 = 0,
\]

as desired.

\end{document}