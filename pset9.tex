\documentclass{article}
\usepackage[utf8]{inputenc}
\usepackage{amsmath, amssymb}
\usepackage{setspace}

\onehalfspacing

\title{Problem Set 9: Independence, Basis, and Dimension}
\author{Tiago C. Botelho}
\date{\today}

\begin{document}

\maketitle

\noindent \textbf{Problem 9.1:} Each of these vectors $\mathbf{v}_i$ is an element of $\mathbb{R}^{4}$, and since $\text{dim }\mathbb{R}^{4} = 4$, there are at most four independent vectors among the $\mathbf{v}_i$. Let's see how many of the vectors actually \textit{are} independent. First, note that:

\begin{align*}
\mathbf{v}_4 = \mathbf{v}_2 - \mathbf{v}_1\\
\mathbf{v}_5 = \mathbf{v}_3 - \mathbf{v}_1\\
\mathbf{v}_6 = \mathbf{v}_3 - \mathbf{v}_2
\end{align*}

Next step is to put the first three vectors as columns of a matrix and apply the elimination algorithm.

\[
A = \begin{bmatrix}
\phantom{-}1 & \phantom{-}1 & \phantom{-}1\\
-1 & \phantom{-}0 & \phantom{-}0\\
\phantom{-}0 & -1 & \phantom{-}0\\
\phantom{-}0 & \phantom{-}0 & -1
\end{bmatrix}
\]

Reduced row echelon form is:

\[
R = \begin{bmatrix}
1 & 0 & 1\\
0 & 1 & 0\\
0 & 0 & 1\\
0 & 0 & 0\\
\end{bmatrix},
\]

so indeed $\mathbf{v}_1, \mathbf{v}_2,$ and $\mathbf{v}_3$ are linearly independent. More than that, we've figured out that the rank of the matrix with all six $\mathbf{v}_i$ for columns is 3.

\noindent \textbf{Problem 9.2:} A plane in $\mathbb{R}^{3}$ can be finitely spanned by two vectors. In particular, we can write for this plane:

\[
\begin{bmatrix}
0\\
0\\
0\\
\end{bmatrix}
=
x\begin{bmatrix}
1\\
0\\
0\\
\end{bmatrix}
+
y\begin{bmatrix}
2\\
0\\
0\\
\end{bmatrix}
+
z\begin{bmatrix}
-3\\
\phantom{-}0\\
\phantom{-}0\\
\end{bmatrix},
\]

and so the plane is the nullspace of the matrix

\[
\begin{bmatrix}
1 & 2 & -3\\
0 & 0 & \phantom{-}0\\
0 & 0 & \phantom{-}0
\end{bmatrix}.
\]

We can already see that two of the variables are free, and only the first one is a pivot. So put $x_2 = 1$ and $x_3 = 0$ to obtain one special solution:

\[
\mathbf{x}_{s}^{1} = \begin{bmatrix}
-2\\
\phantom{-}1\\
\phantom{-}0\\
\end{bmatrix}.
\]

For us to get the other special solution, set $x_2 = 0$ and $x_3 = 0$, and we get:

\[
\mathbf{x}_{s}^{2} = \begin{bmatrix}
3\\
0\\
1\\
\end{bmatrix}.
\]

So a basis for the plane (indeed for the nullspace of that 3 by 3 matrix) is $\{\mathbf{x}_{s}^{1}, \mathbf{x}_{s}^{2}\}$.
\end{document}