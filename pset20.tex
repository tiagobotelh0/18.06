\documentclass{article}
\usepackage[utf8]{inputenc}
\usepackage{amsmath, amssymb}
\usepackage{setspace}

\onehalfspacing

\usepackage[a4paper, total={6in, 8in}]{geometry}

\title{Problem Set 20: Cramer's Rule, Inverse Matrix, and Volume}
\author{Tiago C. Botelho}
\date{\today}

\begin{document}

\maketitle

\noindent \textbf{Problem 20.1:} We only really need to find the cofactors for entries on rows 2 and 3. Let's do it. For row 2, we have:

\begin{align*}
    C_{21} &= -\begin{vmatrix}
    1 & 4\\
    2 & 5\\
    \end{vmatrix}
    = 3 \quad
    C_{22} = \begin{vmatrix}
    1 & 4\\
    1 & 5\\
    \end{vmatrix}
    = 1 \quad
    C_{23} = -\begin{vmatrix}
    1 & 1\\
    1 & 2\\
    \end{vmatrix}
    = -1.
\end{align*}

Whereas for row 3, we have:

\begin{align*}
    C_{31} &= \begin{vmatrix}
    1 & 4\\
    2 & 2\\
    \end{vmatrix}
    = -6 \quad
    C_{32} = -\begin{vmatrix}
    1 & 4\\
    1 & 2\\
    \end{vmatrix}
    = 2 \quad
    C_{33} = \begin{vmatrix}
    1 & 1\\
    1 & 2\\
    \end{vmatrix}
    = 1.
\end{align*}

So the cofactor matrix is:

\[
\begin{bmatrix}
\phantom{-}6 & -3 & \phantom{-}0\\
\phantom{-}3 & \phantom{-}1 & -1\\
-6 & \phantom{-}2 & \phantom{-}1\\
\end{bmatrix}.
\]

We finally multiply $A$ and $C^{T}$ together to obtain:

\[
\begin{bmatrix}
1 & 1 & 4\\
1 & 2 & 2\\
1 & 2 & 5\\
\end{bmatrix}
\begin{bmatrix}
\phantom{-}6 & \phantom{-}3 & -6\\
-3 & \phantom{-}1 & \phantom{-}2\\
\phantom{-}0 & -1 & \phantom{-}1\\
\end{bmatrix}
=
\begin{bmatrix}
3 & 0 & 0\\
0 & 3 & 0\\
0 & 0 & 3\\
\end{bmatrix}
=
3I_3 \implies \det A = 3.
\]

Entry $(1, 3)$ is irrelevant to the determinant because its cofactor is zero, so it can be whatever number and still not change the determinant.

\newpage

\noindent \textbf{Problem 20.2:} First, let us compute each row:

\noindent \textbf{Row 1:}

\[
\partial_{\rho}x = \sin \phi \cos \theta \quad \partial_{\phi}x = \rho \cos \phi \cos \theta \quad \partial_{\theta}x = -\rho \sin \phi \sin \theta
\]

\noindent \textbf{Row 2:}

\[
\partial_{\rho}y = \sin \phi \sin \theta \quad \partial_{\phi}y = \rho \cos \phi \sin \theta \quad \partial_{\theta}y = \rho \sin \phi \cos \theta
\]

\noindent \textbf{Row 3:}

\[
\partial_{\rho}z = \cos \phi \quad \partial_{\phi}z = -\rho \sin \phi \quad \partial_{\theta}z = 0,
\]

So the Jacobian matrix is:

\[
\mathcal{J} = \begin{bmatrix}
\sin \phi \cos \theta & \rho \cos \phi \cos \theta & -\rho \sin \phi \sin \theta\\
\sin \phi \sin \theta & \rho \cos \phi \sin \theta & \rho \sin \phi \cos \theta\\
\cos \phi & -\rho \sin \phi & 0
\end{bmatrix}.
\]

We now set out to compute its determinant. Probably best to choose row three and apply Laplace's theorem.

\begin{align*}
\det \mathcal{J} &= \cos \phi \begin{vmatrix}
\rho \cos \phi \cos \theta & -\rho \sin \phi \sin \theta\\
\rho \cos \phi \sin \theta & \rho \sin \phi \cos \theta\\
\end{vmatrix}
+
\rho \sin \phi \begin{vmatrix}
\sin \phi \cos \theta & -\rho \sin \phi \sin \theta\\
\sin \phi \sin \theta & \rho \sin \phi \cos \theta\\
\end{vmatrix}\\
&= \cos \phi [\rho^{2} \sin\phi\cos\phi \cos^{2}\theta + \rho^{2} \sin\phi\cos\phi\sin^{2}\theta] + \rho\sin\phi [\rho\sin^{2}\phi\cos^{2}\theta + \rho\sin^{2}\phi\sin^{2}\theta]\\
&= \cos\phi[\rho^{2}\sin\phi\cos\phi] + \rho\sin\phi[\rho\sin^{2}\phi]\\
&= \rho^{2}\sin\phi.
\end{align*}
\end{document}