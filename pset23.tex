\documentclass{article}
\usepackage[utf8]{inputenc}
\usepackage{amsmath, amssymb}
\usepackage{setspace}

\onehalfspacing

\title{Problem Set 23: Differential Equations and $\exp(At)$}
\author{Tiago C. Botelho}
\date{\today}

\begin{document}

\maketitle

\noindent \textbf{Problem 23.1:} It's very straightforward:

\[
\frac{\text{d}||\mathbf{u}(t)||^{2}}{\text{d}t} = \frac{\text{d}(u_1^{2} + u_2^{2} + u_3^{2})}{\text{d}t} = 2u_1u_1^{\prime} + 2u_2u_2^{\prime} + 2u_3u_3^{\prime} = (\star).
\]

\[
(\star) = 2[u_1(cu_2 - bu_3) + u_2(au_3 - cu_1) + u_3(bu_1 - au_2)] = 0.
\]

As time goes by, the length $\mathbf{u}$ remains unchanged. This, in turn, implies that: $\mathbf{u}(t)$ must be such that $u_1^{2} + u_2^{2} + u_3^{2} = k \in \mathbb{R}$, for every $t \geq 0$.

\noindent \textbf{Problem 23.2:} First step: we need to find the eigenvalues of $A$, and we do that by solving:

\[
\begin{vmatrix}
1 - \lambda & 1\\
0 & 3 - \lambda
\end{vmatrix}
= 0,
\]

which is equivalent to finding the roots of $p(\lambda) = (1-\lambda)(3-\lambda)$. The spectrum of $A$ is thus $\{1, 3\}$, and we put $\lambda_1 = 1$, $\lambda_2 = 3$. Now, we set out to find the eigenvectors. First, we should find a vector in the nullspace of:

\[
\begin{bmatrix}
0 & 1\\
0 & 2\\
\end{bmatrix}.
\]

Easy: we pick out one unit of column 1 and none of column 2, so an eigenvector is $v_1 = (1, 0)$. Now, we ought to find an element in the nullspace of:

\[
\begin{bmatrix}
-2 & 1\\
\phantom{-}0 & 0\\
\end{bmatrix}
\]

We can pick out one of colum $1$ and $2$ of colum 2 to reach the zero vector. So the eigenvector is $v_2 = (1, 2)$. We can write:

\[
A = S\Lambda S^{-1} = \begin{bmatrix}
1 & 1\\
0 & 2\\
\end{bmatrix}
\begin{bmatrix}
1 & \\
  & 3\\
\end{bmatrix}
\begin{bmatrix}
1 & -0.5\\
0 & \phantom{-}0.5\\
\end{bmatrix}
\]

Now, we compute:

\[
e^{At} = \begin{bmatrix}
1 & 1\\
0 & 2\\
\end{bmatrix}
\begin{bmatrix}
e^{t} & \\
      & e^{3t}\\
\end{bmatrix}
\begin{bmatrix}
1 & -0.5\\
0 & \phantom{-}0.5\\
\end{bmatrix}
=
\begin{bmatrix}
e^{t} & 0.5(e^{3t} - e^{t})\\
0 & e^{3t}\\
\end{bmatrix}.
\]

When $t = 0$, we have:

\[
e^{At} = \begin{bmatrix}
1 & 0.5(1 - 1)\\
0 & 1\\
\end{bmatrix}
=
\begin{bmatrix}
1 & 0\\
0 & 1\\
\end{bmatrix}
=
I.
\]

That's just what we expected! The matrix Taylor expansion evaluates to $I$ when $t = 0$.

Now, let us compute:

\[
\left.\frac{\text{d}e^{At}}{\text{d}t}\right|_{t = 0} = \left.\begin{bmatrix}
e^{t} & 1.5e^{3t} - 0.5e^{t}\\
0 & 3e^{t}\\
\end{bmatrix}\right|_{t=0}
=
\begin{bmatrix}
1 & 1\\
0 & 3\\
\end{bmatrix}
=
A,
\]

again, just what we expected.
\end{document}