\documentclass{article}
\usepackage[utf8]{inputenc}
\usepackage{amsmath}
\usepackage{setspace}

\onehalfspacing

\title{Problem Set 1: Elimination with Matrices}
\author{Tiago C. Botelho}
\date{\today}

\begin{document}

\maketitle

\noindent \textbf{Problem 2.1:} We should subtract three times the first equation from the second equation in the first step of (Gaussian) elimination. Below, we express the system in matrix form:

\[
\begin{bmatrix}
2 & 3\\
6 & 15\\
\end{bmatrix}
\begin{bmatrix}
x\\
y\\
\end{bmatrix}
=
\begin{bmatrix}
5\\
12
\end{bmatrix}.
\]

Now, let us apply elimination. The coefficient matrix becomes:

\[
\begin{bmatrix}
2 & 3\\
0 & 6\\
\end{bmatrix},
\]

so the first and second pivots are 2, and 6, respectively.

We need to apply the same matrix to the right hand side, which gives us:

\[
\begin{bmatrix}
\phantom{-}5\\
-3\\
\end{bmatrix}.
\]

Now, for the final step, we use back substitution, just a fancy name from going bottom to top; we have $6y = -3$, therefore $y = -0.5$; and substituting this in the first equation, we have $2x -1.5 = 5$, \textit{i.e.}, $x = 3.25$.

\noindent \textbf{Problem 2.2:} Want to find a triangular matrix that transforms the Pascal matrix:

\[
\begin{bmatrix}
1 & 0 & 0 & 0\\
1 & 1 & 0 & 0\\
1 & 2 & 1 & 0\\
1 & 3 & 3 & 1\\
\end{bmatrix}
\]

into the simpler, (also) Pascal matrix:

\[
\begin{bmatrix}
1 & 0 & 0 & 0\\
0 & 1 & 0 & 0\\
0 & 1 & 1 & 0\\
0 & 1 & 2 & 1\\
\end{bmatrix}.
\]

Ok, let's think of some elementary matrices that'll do the job when multiplied together. First, need to subtract the first line from all other lines, so we consider:

\[
E = \begin{bmatrix}
\phantom{-}1 & 0 & 0 & 0\\
-1 & 1 & 0 & 0\\
-1 & 0 & 1 & 0\\
-1 & 0 & 0 & 1\\
\end{bmatrix}.
\]

Next up, need to wipe out the 2 in position (3, 2), and actually make it a 1, which we can easily do with row 2; and we can also use row 2 to make entry (4, 2) a 2 instead of a 3. We use:

\[
F = \begin{bmatrix}
1 & \phantom{-}0 & 0 & 0\\
0 & \phantom{-}1 & 0 & 0\\
0 & -1 & 1 & 0\\
0 & -1 & 0 & 1\\
\end{bmatrix}.
\]

One more operation left: we have to subtract what's left of row 3 from what's left of row 4. Easy! We use:

\[
G = \begin{bmatrix}
1 & 0 & \phantom{-}0 & 0\\
0 & 1 & \phantom{-}0 & 0\\
0 & 0 & \phantom{-}1 & 0\\
0 & 0 & -1 & 1\\
\end{bmatrix}
\]

Finally, we compose (multiply together) these matrices to obtain:

\[
T = GFE = \begin{bmatrix}
\phantom{-}1 & \phantom{-}0 & \phantom{-}0 & 0\\
-1 & \phantom{-}1 & \phantom{-}0 & \phantom{-}0\\
\phantom{-}0 & -1 & \phantom{-}1 & \phantom{-}0\\
\phantom{-}0 & \phantom{-}0 & -1 & \phantom{-}1
\end{bmatrix}.
\]

The final calculation was done with the aid of \texttt{numpy}, a Python module for scientific computing.
\end{document}