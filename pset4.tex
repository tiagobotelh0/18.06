\documentclass{article}
\usepackage[utf8]{inputenc}
\usepackage{amsmath}
\usepackage{setspace}

\onehalfspacing

\title{Problem Set 4: Factorization into A = LU}
\author{Tiago C. Botelho}
\date{\today}

\begin{document}

\maketitle

\noindent \textbf{Problem 4.1:} The elimination matrix $E$ can be obtained in a sequential manner, as always. First step is to kill the 2's in rows 2 and 3 of column 1. The elementary matrix that does this is:

\[
E_1 = \begin{bmatrix}
\phantom{-}1 & 0 & 0\\
-2 & 1 & 0\\
-2 & 0 & 1\\
\end{bmatrix},
\]

yielding a transformed version of $A$:

\[
\begin{bmatrix}
1 & \phantom{-}3 & 0\\
0 & -2 & 0\\
0 & -6 & 1\\
\end{bmatrix}.
\]

Final step is to kill off the -6 in position (3, 2). We do that by subtracting three times row 2 from 3, which is equivalent to applying the matrix:

\[
E_2 = \begin{bmatrix}
1 & \phantom{-}0 & 0\\
0 & \phantom{-}1 & 0\\
0 & -3 & 1
\end{bmatrix}
\]

to the transformed version of $A$. The matrix $E$ we were looking for is just:

\[
E = E_2E_1 = \begin{bmatrix}
\phantom{-}1 & \phantom{-}0 & 0\\
-2 & \phantom{-}1 & 0\\
\phantom{-}4 & -3 & 1\\
\end{bmatrix}
\]

We can find $L$ by just keeping a record of the multipliers and undoing what they did to $A$!

\[
L = \begin{bmatrix}
1 & 0 & 0\\
2 & 1 & 0\\
2 & 3 & 1\\
\end{bmatrix}.
\]

Alternatively, one could invert $U$ to find $L = U^{-1}$, but that would be needlessly complicated. We finally obtain our desired expression:

\[
A = LU,
\]

or, in matrix form:

\[
\begin{bmatrix}
1 & 3 & 0\\
2 & 4 & 0\\
2 & 0 & 1\\
\end{bmatrix}
=
\begin{bmatrix}
1 & 0 & 0\\
2 & 1 & 0\\
2 & 3 & 1\\
\end{bmatrix}
\begin{bmatrix}
1 & \phantom{-}3 & 0\\
0 & -2 & 0\\
0 & \phantom{-}0 & 1\\
\end{bmatrix}.
\]

\noindent \textbf{Problem 4.2:} As Jack the Ripper before us, we proceed in a step-by-step manner.

\noindent \textbf{Step 1:} Get rid of the $a$'s in rows 2, 3, and 4 of column 1.

\[
\begin{bmatrix}
\phantom{-}1 & 0 & 0 & 0\\
-1 & 1 & 0 & 0\\
-1 & 0 & 1 & 0\\
-1 & 0 & 0 & 1\\
\end{bmatrix}
\begin{bmatrix}
a & a & a & a\\
a & b & b & b\\
a & b & c & c\\
a & b & c & d\\
\end{bmatrix}
=
\begin{bmatrix}
a & a & a & a\\
0 & b - a & b - a & b - a\\
0 & b - a & c - a & c - a\\
0 & b - a & c - a & d - a\\
\end{bmatrix}.
\]

\noindent \textbf{Step 2:} Kill off the $b - a$ terms in rows 3 and 4 of column 2.

\[
\begin{bmatrix}
1 & \phantom{-}0 & 0 & 0\\
0 & \phantom{-}1 & 0 & 0\\
0 & -1 & 1 & 0\\
0 & -1 & 0 & 1\\
\end{bmatrix}
\begin{bmatrix}
a & a & a & a\\
0 & b - a & b - a & b - a\\
0 & b - a & c - a & c - a\\
0 & b - a & c - a & d - a\\
\end{bmatrix}
=
\begin{bmatrix}
a & a & a & a\\
0 & b - a & b - a & b - a\\
0 & 0 & c - b & c - b\\
0 & 0 & c - b & d - b\\
\end{bmatrix}.
\]

\noindent \textbf{Step 3:} Anihilate the $c - b$ term in row 4 of column 3. This is really the final step.

\[
\begin{bmatrix}
1 & 0 & \phantom{-}0 & 0\\
0 & 1 & \phantom{-}0 & 0\\
0 & 0 & \phantom{-}1 & 0\\
0 & 0 & -1 & 1\\
\end{bmatrix}
\begin{bmatrix}
a & a & a & a\\
0 & b - a & b - a & b - a\\
0 & 0 & c - b & c - b\\
0 & 0 & c - b & d - b\\
\end{bmatrix}
=
\begin{bmatrix}
a & a & a & a\\
0 & b - a & b - a & b - a\\
0 & 0 & c - b & c - b\\
0 & 0 & 0 & d - c\\
\end{bmatrix}.
\]

For us to have $A = LU$ with four pivots, since $L$ will by definition have a diagonal with all 1's, all we need to worry about is the diagonal of $U$, namely, we need to make sure that:

\[
a \cdot (b - a) \cdot (c - b) \cdot (d - c) \neq 0.
\]

A less elegant way of putting it would be to impose the four conditions $d \neq c \neq b \neq a \neq 0$.

\end{document}