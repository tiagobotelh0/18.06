\documentclass{article}
\usepackage[utf8]{inputenc}
\usepackage{amsmath, amssymb}
\usepackage{setspace}

\onehalfspacing

\title{Problem Set 11: Matrix Spaces; Rank 1; Small World Graphs}
\author{Tiago C. Botelho}
\date{\today}

\begin{document}

\maketitle

\noindent \textbf{Problem 11.1:} We can write $I_3$, the 3 by 3 identity matrix as:

\[
\begin{bmatrix}
1 & 0 & 0\\
0 & 0 & 1\\
0 & 1 & 0\\
\end{bmatrix}
+
\begin{bmatrix}
0 & 1 & 0\\
1 & 0 & 0\\
0 & 0 & 1\\
\end{bmatrix}
-
\begin{bmatrix}
0 & 1 & 0\\
0 & 0 & 1\\
1 & 0 & 0\\
\end{bmatrix}
-
\begin{bmatrix}
0 & 0 & 1\\
1 & 0 & 0\\
0 & 1 & 0\\
\end{bmatrix}
+
\begin{bmatrix}
0 & 0 & 1\\
0 & 1 & 0\\
1 & 0 & 0\\
\end{bmatrix}.
\]

Now, assume for the sake of contradiction that we can find nonzero constants $c_1, ..., c_5$ such that $\sum_{i=1}^{5}c_i P_i = \mathbf{0}$. Then:

\[
\begin{bmatrix}
c_1 & c_2 + c_3 & c_4 + c_5\\
c_2 + c_4 & c_5 & c_1 + c_3\\
c_3 + c_5 & c_1 + c_4 & c_2\\
\end{bmatrix}
=
\begin{bmatrix}
0 & 0 & 0\\
0 & 0 & 0\\
0 & 0 & 0\\
\end{bmatrix}.
\]

This, in turn, implies that $c_1 = c_5 = c_2 = 0$ (diagonal); which implies that $c_3 = 0$ (entry (3, 1)), and $c_4 = 0$ (entry (3, 2)), \textit{i.e.}, $c_i = 0$ for all $i \in \{1, ..., 5\}$, contrary to our choice of the $c_i$.

\noindent \textbf{Problem 11.2:}

\noindent \textbf{(a)} A quick mental procedure tells us that $\text{dim }\mathcal{N}(A) = 1$, so $[1 \ 1 \ 1]^{T}$ forms a basis for the nullspace of $A$. Therefore, for $AX$ to be equal to $\mathbf{0}$, we need the columns $X$ to be multiples of $[1 \ 1 \ 1]^{T}$.

\noindent \textbf{(b)} First, note that:

\[
\begin{bmatrix}
\phantom{-}1 & \phantom{-}0 & -1\\
-1 & \phantom{-}1 & \phantom{-}0\\
\phantom{-}0 & -1 & \phantom{-}1\\
\end{bmatrix}
\begin{bmatrix}
a & b & c\\
d & e & f\\
g & h & i\\
\end{bmatrix}
=
\begin{bmatrix}
a - g & b - h & c - i\\
d - a & e - b & f - c\\
g - d & h - e & i - f\\
\end{bmatrix},
\]

so these matrices all have one important property: the elements along any of the columns add up to zero! So matrices of this type can be written as:

\[
\begin{bmatrix}
a & b & c\\
a' & b' & c'\\
-(a + a') & -(b + b') & -(c + c')\\
\end{bmatrix}.
\]

\noindent \textbf{(c)} The dimension of $\mathcal{N}(A)$ is 1, as we have already said before; and the dimension of $\mathcal{C}(A)$ is 2. The reason why they add up to $n$, the number of columns, is that we can use the pivot columns as a basis for $\mathcal{C}(A)$ and the remaining columns -- the free variable columns -- as a basis for $\mathcal{N}(A)$.
\end{document}