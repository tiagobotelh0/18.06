\documentclass{article}
\usepackage[utf8]{inputenc}
\usepackage{amsmath, amssymb}

\makeatletter
\renewcommand*\env@matrix[1][*\c@MaxMatrixCols c]{%
  \hskip -\arraycolsep
  \let\@ifnextchar\new@ifnextchar
  \array{#1}}
\makeatother

\usepackage{setspace}

\onehalfspacing

\title{Exam 1}
\author{Tiago C. Botelho}
\date{\today}

\begin{document}

\maketitle

\noindent \textbf{Exercise 1.}

\noindent \textbf{(a)} $A$ is $m$ by $n$; for us to be able to multiply $A$ by $x$, we need $x$ to have $n$ rows; and since the result is an element of $\mathbb{R}^{3}$, we know that $m = 3$.

Because the system of linear equations:

\[
A\mathbf{x} = \begin{bmatrix}
1\\
1\\
1\\
\end{bmatrix}
\]

has no solutions, we know that the columns of $A$ do \textit{not} span all of $\mathbb{R}^{3}$, so $n$ is at most 2. On the other hand, the system

\[
A\mathbf{x} = \begin{bmatrix}
0\\
1\\
0\\
\end{bmatrix}
\]

has a unique solution, which means that at least one of the columns of $A$ is independent of the others, meaning $n$ is at least 1. Thus $n \in \{1, 2\}$.

Finally, because $\text{rank }A \leq \min\{m, n\}$, it follows that $\text{rank }A \leq 2.$ Of course, the fact that the second system has a unique solution means that $r = n$, so $r \in \{1, 2\}$.

\noindent \textbf{(b)} Well, the fact that 
\[
A\mathbf{x} = \begin{bmatrix}
0\\
1\\
0\\
\end{bmatrix}
\]

has a \textit{unique} solution, we know that $\mathcal{N}(A) = \{0\}$ (otherwise we could add multiples of a vector in the basis of $\mathcal{N}(A)$ to obtain infinitely many other solutions). Therefore the only solution to $A\mathbf{x = 0}$ is the zero vector itself.

\noindent \textbf{(c)} We can take:

\[
A = \begin{bmatrix}
0\\
1\\
0\\
\end{bmatrix}.
\]

And it satisfies all of our criteria: first, it has 3 rows and $n = 1$ column(s). The first system will have no solution because this vector spans a line in $\mathbb{R}^{3}$ that does not contain $(1, 1, 1)$. The second system has a unique solution with $x = 1$.

\noindent \textbf{Exercise 2.}

\noindent \textbf{(a)} If $A$ is reduced to $I_3$ when we apply elimination, then $E_{23}E_{31}E_{21}A = I_3$, which means that $E_{23}E_{31}E_{21} = A^{-1}$. Therefore:

\[
A^{-1}
=
\begin{bmatrix}
1 & 0 & \phantom{-}0\\
0 & 1 & -1\\
0 & 0 & \phantom{-}1\\
\end{bmatrix}
\begin{bmatrix}
\phantom{-}1 & 0 & 0\\
\phantom{-}0 & 1 & 0\\
-3 & 0 & 0\\
\end{bmatrix}
\begin{bmatrix}
\phantom{-}1 & 0 & 0\\
-4 & 1 & 0\\
\phantom{-}0 & 0 & 1\\
\end{bmatrix}
=
\begin{bmatrix}
\phantom{-}1 & 0 & \phantom{-}0\\
-1  & 1 & -1\\
-3  & 0  & \phantom{-}1\\
\end{bmatrix}.
\]

\noindent \textbf{(b)} Well, if $E_{23}E_{31}E_{21} = A^{-1}$, then clearly $(E_{23}E_{31}E_{21})^{-1} = (A^{-1})^{-1} = A$. So the matrix we're really after is $E_{21}^{-1}E_{31}^{-1}E_{23}^{-1}$. Each of these matrices does the opposite row operation of their inverse, so:

\[
A =
\begin{bmatrix}
1 & 0 & 0\\
4 & 1 & 0\\
0 & 0 & 1\\
\end{bmatrix}
\begin{bmatrix}
1 & 0 & 0\\
0 & 1 & 0\\
3 & 0 & 0\\
\end{bmatrix}
\begin{bmatrix}
1 & 0 & 0\\
0 & 1 & 1\\
0 & 0 & 1\\
\end{bmatrix}
=
\begin{bmatrix}
1 & 0 & 0\\
4 & 1 & 1\\
3 & 0 & 0\\
\end{bmatrix}
\]

\noindent \textbf{(b)} The lower triangular in $A$ does the opposite of the first and second operations all at once, but not the third (which is going further than just simple elimination). So:

\[
L =
\begin{bmatrix}
1 & 0 & 0\\
4 & 1 & 0\\
3 & 0 & 0\\
\end{bmatrix}.
\]

Where we understand that:

\[
U =
\begin{bmatrix}
1 & 0 & 0\\
0 & 1 & 1\\
0 & 0 & 1\\
\end{bmatrix}.
\]

\noindent \textbf{Exercise 3.} For this exercise, a useful thing to do is to row reduce the augmented matrix $A$ with $(1, c, 0)$ tacked onto it.

\noindent \textbf{Step 1:} Subtract 3 (row 1) from row 2; we get:

\[
\begin{bmatrix}[cccc|c]
1 & 1 & \phantom{-}2 & \phantom{-}4 & 1\\
0 & (c-3) & -4 & -4 & (c-3)\\
0 & 0 & \phantom{-}2 & \phantom{-}2 & 0\\
\end{bmatrix}.
\]

\noindent \textbf{Step 2:} Divide row 3 by 2, and then add 4 (row 3) to row 2; we get:

\[
\begin{bmatrix}[cccc|c]
1 & 1 & 2 & 4 & 1\\
0 & (c-3) & 0 & 0 & (c-3)\\
0 & 0 & 1 & 1 & 0\\
\end{bmatrix}.
\]

Since we do not know whether $c$ is equal to 3 or not, we need to proceed carefully.

\noindent \textbf{Step 3:} We assume that $c = 3$. That being the case, the row reduced echelon form of the augmented matrix can be obtained if we swap rows 2 and 3 and then subtract 2 (row 2) from row 1:

\[
\begin{bmatrix}[cccc|c]
1 & 1 & 0 & 2 & 1\\
0 & 0 & 1 & 1 & 0\\
0 & 0 & 0 & 0 & 0\\
\end{bmatrix}.
\]

\noindent \textbf{Step 3$^{\prime}$:} We assume that $c \neq 3$. That being the case, the row reduced echelon form of the augmented matrix can be obtained if we divide row 2 by $(c-3) \neq 0$, and then subtract (row 2) from row 1 as well as 2 row(3) from row 1; we get:

\[
\begin{bmatrix}[cccc|c]
1 & 0 & 0 & 2 & 0\\
0 & 1 & 0 & 0 & 1\\
0 & 0 & 1 & 1 & 0\\
\end{bmatrix}.
\]

\noindent \textbf{(a)} We split the answer into two separate cases:

If $c = 3$, then columns 1 and 3 have pivots in them, so a basis for $\mathcal{C}(A)$ is the set $\{(1, 3, 0), (2, 2, 2)\}$.

If $c \neq 3$, then columns 1, 2, and 3 have pivots in them, so a basis for $\mathcal{C}(A)$ is the set $\{(1, 3, 0), (1, c, 0), (2, 2, 2)\}$.

\noindent \textbf{(b)} A basis for the nullspace of $A$ can be determined if we find the special solutions to $A\mathbf{x = 0}$. Again, two possibilities arise:

If $c = 3$, we have two free variables, namely, $x_2$ and $x_4$. First, set $x_2 = 1$ and $x_4 = 0$ to find the first special solution:

\[
\begin{bmatrix}
-1\\
\phantom{-}1\\
\phantom{-}0\\
\phantom{-}0\\
\end{bmatrix};
\]

and then, set $x_2 = 0$ and $x_4 = 1$ to find the second special solution:

\[
\begin{bmatrix}
-2\\
\phantom{-}0\\
-1\\
\phantom{-}1\\
\end{bmatrix}.
\]

Therefore, a basis for $\mathcal{N}(A)$ is the set: $\{(-1, 1, 0, 0), (-2, 0, -1, 1)\}$.

On the other hand, if $c \neq 3$, we have a single free variable, namely, $x_4$. So only one special solution is needed. Set $x_4 = 1$ to obtain:

\[
\begin{bmatrix}
-2\\
\phantom{-}0\\
-1\\
\phantom{-}1\\
\end{bmatrix}.
\]

So in this case, a basis for for $\mathcal{N}(A)$ is the singleton: $\{(-2, 0, -1, 1)\}$.

\noindent \textbf{(c)} Easy. A particular solution is $(0, 1, 0, 0)$. This is just the vector that picks out column 2, \textit{i.e.}, exactly the right hand side!

So a complete solution can be one of two things. If $c = 3$, a complete solution is $(0, 1, 0, 0) + \alpha (-1, 1, 0, 0) + \alpha^{\prime} (-2, 0, -1, 1)$, where $\alpha \in \mathbb{R}$ and $\alpha^{\prime} \in \mathbb{R}$. On the other hand, if $c \neq 3$, a complete solution is $(0, 1, 0, 0) + \alpha (-2, 0, -1, 1)$, where $\alpha \in \mathbb{R}$.

\noindent \textbf{Exercise 4.}

\noindent \textbf{(a)} If $A$ is 3 by 5, then $\text{dim }\mathcal{N}(A)$ is bounded above by 5 (for a vector to be in the nullspace, it must have 5 components); and bounded below by $2$ (think of it as 5 variables and 3 equations). What we're saying is $\text{dim }\mathcal{N}(A) \in \{2, 3, 4, 5\}$.

\noindent \textbf{(b)} We have three pivots, and they lie on columns 1, 4, and 5, so the column space is spanned by columns 1, 4, and 5. Column 2 is four times column 1; and column 3 is just the zero vector of $\mathbb{R}^{3}$.

\noindent \textbf{(c)} Well, for invertible matrices, $R = I_3$; otherwise, for singular matrices, $R$ might have nonzero entries above the diagonal. We can combine these two facts to conclude that all possible row reduced echelon forms span the subspace of all upper triangular matrices.
\end{document}