\documentclass{article}
\usepackage[utf8]{inputenc}
\usepackage{amsmath, amssymb}
\usepackage{setspace}

\onehalfspacing

\title{Problem Set 15: Projections onto Subspaces}
\author{Tiago C. Botelho}
\date{\today}

\begin{document}

\maketitle

\noindent \textbf{Problem 15.1:} The problem tells us that:

\[
A = \begin{bmatrix}
1 & 0 & 0\\
0 & 1 & 0\\
0 & 0 & 1\\
0 & 0 & 0\\
\end{bmatrix}.
\]

$\mathcal{C}(A)$, the column space of $A$, is just the set of vectors spanned by $(1, 0, 0, 0)$, $(0, 1, 0, 0)$, and $(0, 0, 1, 0)$, since all three columns are linearly independent. So if we project $\mathbf{b}$ onto $\mathcal{C}(A)$, we're losing all the data we had in the fourth dimension, \textit{i.e.}, the projection is just $(1, 2, 3, 0)$. We can thus think of the projection matrix as:

\[
\begin{bmatrix}
1 & 0 & 0 & 0\\
0 & 1 & 0 & 0\\
0 & 0 & 1 & 0\\
0 & 0 & 0 & 0\\
\end{bmatrix},
\]

which is 4 by 4.

\noindent \textbf{Problem 15.2} Assuming $P^{2} = P$, let's expand the form we're interested in. We have:

\begin{align*}
(I - P)^{2} &= (I - P)(I - P)\\
&= I^{2} - PI - IP + P^{2}\\
&= I - P - P + P\\
&= I - P.
\end{align*}

Note that $I - P$ is the matrix:

\[
\begin{bmatrix}
0 & 0 & 0 & 0\\
0 & 0 & 0 & 0\\
0 & 0 & 0 & 0\\
0 & 0 & 0 & 1\\
\end{bmatrix},
\]

so it's projecting onto the orthogonal complement of the column space of $A$, \textit{i.e.}, it's projecting onto the \textbf{left nullspace} of $A$.
\end{document}